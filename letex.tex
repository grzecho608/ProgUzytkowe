\documentclass[a4paper,12pt]{article}
\usepackage[MeX]{polski}
\usepackage[utf8]{inputenc}

%opening
\title{wzory matematyczne}
\author{Grzegorz Sokołowski}

\begin{document}
$$\lim_{n\to\infty} \sum_{k=1}^{n} \ \frac{1}{k^2}  = \frac{\pi^2}{6} $$
\label{11}

$$\prod_{i=2}^{n=i^2} \ = \frac{ \lim^{n \to 4} (1+ \frac{1}{n})}{\sum k(\frac{1}{n}) 
$$
\label{12}
Łatwo równanie 11 jest doprowadzić do 12
$$\int_{2}^{infty} \frac{1}{log_2 x} dx= \frac{1}{x}\sin x=1 - \cos^2(x)
 $$ \label{13}

\begin{equation} 
\left[ \begin{array}{cccc}
a-{11} & a_{12} & \ldots & a-{1k}\\
a-{21} & a_{22} & \ldots & a-{2k}\\
\vdots & \vdots & \ddots & \vdots	\\
a-{k1} & a_{k2} & \ldots & a_{kk} \\
\end{array} \right]
\end{equation}

*

\begin{equation} 
\left[ \begin{array}{c}
x-{1}
x-{2}
\vdots
x-{k}
\end{array} \right]
\end{equation}

=

\begin{equation} 
\left[ \begin{array}{c}
b-{1}
b-{2}
\vdots
b-{k}
\end{array} \right]
\end{equation}
\label{14}

$\big( a_{1}=a_{1}(x)\big) \wedge \big( a_{2}=a_{2}(x)\big)  \wedge \ldots \wedge $
$\big( a_{k}=a_{k}(x)\big) \rightarrow \big( d=d(u)\big) $
\label{15}


\maketitle

\begin{abstract}

\end{abstract}

\section{}

\end{document}