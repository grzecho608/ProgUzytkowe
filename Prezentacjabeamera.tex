\documentclass[]{beamer}
\usepackage[MeX]{polski}
\usepackage[utf8]{inputenc}
\usepackage{polski}
\beamersetaveragebackground{blue!10}
\usetheme{Warsaw}
\usepackage{beamerthemesplit}


%opening
\title{Prezentacja}
\author{Grzegorz Sokołowski}

\begin{document}
\maketitle
\frame
{
\frametitle{AWO}
\begin{itemize}
\item Agenturalny Wywiad Operacyjny – tajna komórka wywiadowcza powstała w latach 60 XX wieku, 
\item część struktury Zarządu II Sztabu Generalnego Wojska Polskiego, czyli wywiadu wojskowego PRL.
\item Powołanie AWO zarządził Sztab Generalny WP w 1966, w lipcu 1967 roku powstał Oddział Agenturalnego Wywiadu Operacyjnego (w oparciu o Oddział X).
\end{itemize}
}


\frame
{
\frametitle{Cele}
\begin{itemize}
\item Celem AWO było zdobywanie, dla potrzeb dowództw szczebla operacyjnego, przede wszystkim w okresie wojny, wiadomości i materiałów wywiadowczych, dotyczących sił zbrojnych i przygotowań wojennych potencjalnych przeciwników na obszarze przewidywanych działań bojowych wojsk własnych
\end{itemize}
}


\frame
{
\frametitle{Decyzja}
\begin{itemize}
\item Decyzja o powołaniu Agenturalnego Wywiadu Operacyjnego została prawdopodobnie podjęta w Moskwie i zaakceptowana przez kierującego Sztabem Generalnym WP,
\item Późniejszego Ministra Obrony Narodowej generała armii Wojciecha Jaruzelskiego.
\item AWO miał działać w czasie „W” bezpośrednio w strukturze wywiadowczej Zjednoczonych Sił Zbrojnych Układu Warszawskiego.
\end{itemize}
}

\frame
{
\frametitle{Zadanie}
\begin{itemize}
\item Głównymi zadaniami AWO miało być werbowanie i szkolenie szpiegów oraz rezydentów wywiadu.
\end{itemize}
}


\frame
{
\frametitle{Informacje}
\begin{itemize}
\item W czasie pokoju AWO zbierał informacje o siłach zbrojnych przeciwnika w potencjalnym konflikcie,
\item o dowódcach wojskowych oraz związkach operacyjnych i taktycznych państw NATO,
\item a także wojskowych ośrodkach naukowo-badawczych, zakładach przemysłowych itp.
\end{itemize}
}



\frame
{
\frametitle{Werbunek}
\begin{itemize}
\item  Prowadził także pracę werbunkową, dzięki której w krajach zachodnich miała zostać utworzona sieć agenturalna.
\item Podczas wojny AWO miał zająć się agenturalną pracą na tyłach przeciwnika.
\item Mieli też zostać przeszkoleni rezydenci, którzy na wypadek wojny mieli zostać przerzuceni na tyły wroga i tam nadzorować siatkę agentów wywiadu.
\end{itemize}
}



\frame
{
\frametitle{Zastępca Szefa Zarządu}
\begin{itemize}
\item Pod koniec lat 70 XX w. i początku 80 XX wieku Oddział XIII Zarządu II SG WP (Agenturalny Wywiad Operacyjny) podlegał Zastępcy Szefa Zarządu II SG WP ds. Operacyjnych.
\end{itemize}
}



\frame
{
\frametitle{Reorganizacja}
\begin{itemize}
\item W latach 1983–1984 Zarząd II SG WP przeszedł wielką reorganizację. AWO został podporządkowany Agenturalnemu Wywiadowi Strategicznemu a ten podlegał pod tzw. Aparat Krajowy Zarządu II Sztabu Generalnego Wojska Polskiego.
\end{itemize}
}



\frame
{
\frametitle{Czy AWO istnieje?}
\begin{itemize}
\item Niewiele wiadomo o rzeczywistej działalności AWO, jej istnienie potwierdził gen. broni Czesław Kiszczak – szef wywiadu wojskowego w latach 70 XX wieku.
\item Informacja o istnieniu tej komórki została ujawniona dopiero w trakcje procesu lustracyjnego Józefa Oleksego, któremu rzecznik interesu publicznego zarzucił tajną współpracę z AWO (potwierdzoną przez Sąd Lustracyjny w pierwszej instancji).
\end{itemize}
}



\frame
{
\frametitle{AWO jako program?}
\begin{itemize}
\item AWO należy uznać za odmianę programów typu stay-behind, tworzonych przez służby specjalne państw NATO i Układu Warszawskiego na wypadek wybuchu wojny.
\end{itemize}
}







\end{document}